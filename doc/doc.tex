\documentclass[fleqn,english,openany]{scrbook}
\usepackage[utf8]{inputenc}
\usepackage[T1]{fontenc}
\usepackage[english]{babel}
\usepackage{fancyhdr,afterpage}

\usepackage{hyperref}

\setcounter{secnumdepth}{3}
\setcounter{tocdepth}{3}

\setlength{\oddsidemargin}{0mm}
\setlength{\evensidemargin}{0mm}
\setlength{\textwidth}{159.2mm}

\date{}
\begin{document}

\pagestyle{empty}

\setlength{\parindent}{0mm}

\setcounter{page}{2}

\begin{titlepage}
\begin{center}

{\Huge xgbcemu} \vspace{10mm}

{\large {\bf Documentation}}

\end{center}

\newpage

{\bf xgbcemu -- Documentation} \\
by Hanna Reitz \vspace{5mm}

Copyright {\textcopyright} Hanna Reitz, 2010. Some rights reserved (GFDL). \\
Not printed in Germany. \vspace{5mm}

Published by XanClic. \vspace{5mm}

Yocourse books may be purchased for every use except burning it. But there is no need to buy them because there is always
a free online edition available (URL varies). \vspace{5mm}

\begin{tabbing}
{\large \bf Word processor}..... \= \kill
{\large \bf Editor:} \> Hanna Reitz \\ \\
{\large \bf Word processor:} \> \LaTeX \\ \\
{\large \bf Printing History:} \\
\hspace{5mm} April 2010: \> First Edition.
\end{tabbing}

\vspace{10mm}

While every precaution has been taken in the preparation of this book, the publisher and the authors assume no
responsibility for errors or omissions, or for damages resulting from the use of the information contained herein.

\end{titlepage}

\pagestyle{fancy}

\fancyhf{}

\renewcommand{\headrulewidth}{0.5pt}
\renewcommand{\footrulewidth}{0pt}
\newcommand{\currentchaptername}{}
\newcommand{\currentsectionname}{}
\newcommand{\currentsubsectionname}{}

\renewcommand{\chaptermark}[1]
{
    \label{sec:#1}
    \renewcommand{\currentchaptername}{#1}
    \thispagestyle{empty}
    \rhead[Chapter \thechapter]{\thepage}
    \lhead[\thepage]{#1}
}

\renewcommand{\sectionmark}[1]
{
    \label{sec:\currentchaptername:#1}
    \renewcommand{\currentsectionname}{#1}
}

\renewcommand{\subsectionmark}[1]
{
    \label{sec:\currentchaptername:\currentsectionname:#1}
    \renewcommand{\currentsubsectionname}{#1}
}

\renewcommand{\subsubsectionmark}[1]
{
    \label{sec:\currentchaptername:\currentsectionname:\currentsubsectionname:#1}
}

\lhead[\roman{page}]{}
\rhead[]{\roman{page}}

\makeatletter
\let\ps@plain\ps@fancy
\makeatother

\tableofcontents


% \setlength{\parskip}{1mm}
% \setlength{\parindent}{4mm}

\ifodd\thepage
    \newpage

    ~
\fi

\newpage

\setcounter{page}{1}

\lhead[\thepage]{}
\rhead[]{\thepage}


\chapter{Introduction}

\section{Goal}

xgbcemu is a free Game Boy Color emulator. The project's goal is to provide a portable emulator, “portable” by means of
being portable between different operating systems, not necessarily between different platforms. To achieve that, most
of the os-dependend functions have been moved to a special “os-dep” folder. Those functions are creating a drawing area,
drawing on it and connecting to a remote host, as well as just putting a string on the current console (similar to a
“printf”). In order to port xgbcemu to another OS, it should be sufficient to create a new subfolder in “os-dep” which
contains the OS specific functions.


\section{What is it?}

As said before, xgbcemu is a Game Boy Color emulator. It emulates a GBC's CPU (Z80-like), its display mechanism, the link
cable, etc. (sound is not emulated, because that would not be portable at all---it would already be hard to find one
working solution for Linux).



\chapter{Using xgbcemu}

\section{Controls}

xgbcemu offers the buttons a GBC has: A, B, Start, Select and of course Up, Down, Left and Right. However, how to press
those buttons differs on every OS.

\subsection{Linux/SDL}

With SDL, the keyboard mapping is as follows:

\begin{itemize}
\item A is pressed by pressing the according key, same applies to B, Up, Down, Left and Right (using the arrows).
\item Select is pressed by using S.
\item Start is pressed by using the return key.
\item In order to save the battery content manually, press the space key.
\item To toggle full speed mode, use the left shift key.
\end{itemize}
(Battery content and full speed mode are discussed later on)

To quit the program, simply close the window.

\subsection{CDI-13h}

This mode is for operating systems which support the CDI/BIOS driver interface (i.e., a vm86 function). Though it might
work on tyndur only, the mapping is as follows:

\begin{itemize}
\item A is pressed by pressing the according key, same applies to B.
\item Up, Down, Left and Right are used with I, K, J and L, respectively.
\item Select is pressed by using S.
\item Start is pressed by using the return key.
\item In order to save the battery content, press the space key.
\end{itemize}

In order to quit the program (which is in full-screen mode here), press Q.


\section{Shell}

The shell is a mode to enter special commands which cannot be entered by pressing some keys. Though it is theoretically
present on every OS, there's just Linux (for now) which offers a keystroke to enter it (which is Ctrl-C).

When the shell is entered, the emulation will break and a prompt “xgbcemu\$” will appear. You may now enter one of the
following commands.

\subsection{exit}

“exit” is equal to “quit” and exits the shell, continuing the emulation.

\subsection{kill}

This command will kill xgbcemu, which is useful if the program hangs.

\subsection{change time}

Some GBC cartridges feature real time clocks, which are emulated by xgbcemu. The time the clock returns to the emulated
program can be changed with the command “change\_time”. You may enter a new day of year and the current time. The
settings will be reset upon exiting xgbcemu.

\subsection{link}

This command is for manipulating the link cable. See the according section for details.


\section{Speed}

xgbcemu running on a relatively modern computer with Linux/SDL should be a lot faster than a real GBC, which is actually
annoying. To prevent that, there is a timer which tries to remove the differences (and also CPU load). That timer relies
(on Linux) on a special x86 CPU instruction, called RDTSC. That instruction returns the value of an internal high
resolution timer---the problem is that this resolution is unknown and different on every computer, thus its frequency has
to be determined, which is nearly impossible without having full control over the CPU. The only program which might be
capable of doing so is the OS kernel (i.e., Linux itself). Thus, xgbcemu might try to determine the TSC frequency (TSC
means “time stamp counter”, RDTSC means “read TSC”), but the result would not be exact.

The TSC frequency should be equal to the CPU frequency, which can be read from /proc/cpuinfo. xgbcemu does that, but the
result is obviously still not perfect---the emulator is still running too fast.

\subsection{Full speed mode}

Sometimes you might want to disable the “brake” and run xgbcemu at full speed. This is done by enabling the full speed
mode (on Linux: press left shift to enable it, another time to disable it again).


\section{Link cable}

The link cable is emulated via establishing a TCP/IP connection. Every xgbcemu instance tries to create a TCP server on
port 4224. If you want to connect to another computer (“plug in the link cable”), enter the shell and use the “link plug”
command. It takes one parameter, the destination host. If you have to instances running on the same computer and want to
connect both, use “link plug localhost” or “link plug 127.0.0.1” on the instance which was unable to create a server
(because the other one already created one on port 4224).

But pay attention: The link support is experimental and not expected to work (which is due to heavy differences in design
between the simple GBC link protocol and the sophisticated TCP protocol)! Whenever an action might be used for
synchronisation (i.e., if both of the emulated programs wait for another program to connect), try to begin with that
action on the “connection master”, which is the server (i.e., the instance where “link plug” was {\em not} used). If it
did not work: Try again.

To unplug a virtual link cable (i.e., to close the TCP connection), use the “link unplug” shell command (which takes no
parameter). To display the current status (whether a link cable has been plugged in or not), use “link status”.

\end{document}
